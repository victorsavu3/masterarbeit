\chapter{Conclusion}

\section {Related Work}

\subsection{Profiling and Tracing}

The tools presented in this thesis are able to gather tracing information about the application, but it is imprecise and slow in this regard. As such it is recommended to use an existing profiler or tracer to discover the performance bottlenecks. Our framework becomes useful after this step, as it helps the user to determine how to eliminate the bottlenecks.

\subsection{Parceive}

Parceive \cite{parceive} is a tracing-based tool for parallelizing existing sequential software. It focuses on high-level architecture analysis, but also gathers fine-grained memory accesses and call information.

The implementation of this tool generates a database compatible with Parceive, keeping all of its existing views functional. The biggest difference is that additional information about tags is recorded, making our analysis possible. The tracing can also be controlled using tags, making it possible for large applications to be analyzed.

We have also created multiple new views that are designed to help a developer in analyzing parallelization opportunities. These visualizations are based on the framework provided by the Parceive UI.

Parceive also contains functionality that is lacking in our tool such as:

\begin{itemize}
	\item Windows compatibility
	\item Loop detection
	\item Variable name resolution
\end{itemize}

\subsection{Intel Advisor}

Intel Advisor \cite{inteladvisor} is a collection of related tools that focus on helping developers parallelize applications to take advantage of modern processors. It focuses on vectorization and thread parallelism.

The section analysis that our framework provides is also possible in Intel Advisor, but with a different workflow. The source code of the application needs to be annotated using macros provided by the tool such as \texttt{ANNOTATE\_SITE\_BEGIN} or \texttt{ANNOTATE\_TASK\_BEGIN}. This requires rebuilding the application and then instrumenting it using Intel Pin. Our tool is able to instrument the application without the need for source code changes, reducing the number of steps required to gain information about the application.

Intel Advisor offers many features not available in our tool such as vectorization and memory access pattern analysis. We expect the two tools to complement each other when trying to achieve the greatest performance gains.

\section{Future Work}

Even though our framework is useful in their current state, further work needs to be performed to improve its versatility and usability.

\subsection{Memory reference identification}

Currently our tools identify memory location by as being part of the stack, global memory or the heap. For each of these cases more details can be provided to users by using the debugging information present in applications.

\begin{itemize}
	\item \textbf{Stack references} are almost always part of an variable. We can display the name of the variable instead of its memory location. For complex types such as arrays and structures we have sufficient information to display the member accessed.
	
	\item \textbf{Global references} are always named. Similarly to stack variables we can display their name and the accessed member.
	
	\item \textbf{Heap references} are the most difficult case. They are only identified by their name and size. A generic solution for these references does not exist, but in some situations we can identify the type that the user has specified for the reference. With this information we can also identify the specific member accessed in arrays or structures. 
\end{itemize}

Parts of this functionality have already been integrated into Parceive, but obtaining these details would require our tools to also parse and process the \texttt{DWARF} information embedded in an application. This implementation will take time to develop and will incur a small performance overhead, but will make our tools easier to use.

\subsection{Increase dynamic analysis performance}

Our tools use \texttt{PIN\_GetSource\_Location} for each instruction to determine the structure of the application. Tagging only needs the addresses of very specific source locations and we could dramatically increase performance by accessing \texttt{DWARF} information directly. This allows the dynamic analysis to find the relevant instructions addresses directly.

\subsection{Loop detection}

Parceive performs loop detection and can provide this information to visualization. In order to improve the quality of our output we would also need to perform loop detection.

Loop detection could also become useful for tagging code. Users will be able to insert tags at the start and end of the loop and its iteration directly, without relying on lines of code.

\subsection{Control dependency detection}

We currently consider only data dependencies, as described in Section \ref{cap2:dependencies}, and ignore control dependencies for conflict detection. Intel Pin provides functionality to intercept all system calls made by an application and we can use this feature to detect  control dependencies.

This feature will require us to accurately model the effects of system calls and to define when a conflict occurs if their ordering is changed. We are not aware of any tool that provides this functionality, but symbolic execution engines such as Klee \cite{klee} have fully modeled system calls.

\subsection{Other parallelization models}

For this thesis we have implemented section and pipeline analysis, but it is easy to extend our tools to other parallelism models. Further development can enable the analysis of nested parallel patterns, reductions and automatic ordering of tasks performing pure computations based on data dependencies.

\subsection{\texttt{pintool\_dynamic.so} analysis hooks}

The easiest way currently to develop new tools is to process the resulting database. This is sufficient for many analyses, but the processing time required might be excessive.

Both Section and Pipeline analyses perform computation while the application is being instrumented. This allows them to use the current state of the application to reduce the amount of processing required. New tools that need this kind of optimization will need to be integrated into \texttt{pintool\_dynamic.so}.

It is possible to implement hooks into \texttt{pintool\_dynamic.so} to allow analyses to be developed independently from the main library. This will considerably reduce the difficulty of developing new tools.

\section{Conclusion}

Traditional tools such as debuggers have focused on following a single thread of execution in isolation. With the increasing demand for more parallelization of software we believe that more tools will be required to aid developers in their work. This will allow applications to be written faster and with less unexpected hazards.

In this thesis we have presented our dynamic binary instrumentation tools designed to help in parallelization endeavors for existing applications. With their help a user can easily determine to which parts of an application to invest his refactoring efforts. They also highlight the data dependencies between sections of code, decreasing the likelihood of introducing new bugs.

The framework that we have developed also allows additional tools to be implemented in order to provide new insights into applications. We hope that we and others will create innovative tools to help future developers in their work.