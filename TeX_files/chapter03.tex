\chapter{Implementation}

\section{Introduction}

\subsection{Tools}
\subsection{Analysis}
\subsection{Visualizations}
\subsection{Workflow}
\subsection{Architecture}

\section{Concepts}

\subsection{Filtering}

\subsection{Tags}

\begin{itemize}
	\item [Section] marks the boundaries of a section of code that can be executed in parallel.
	\item [SectionTask] is a task inside a section. Each task can be run in parallel to any other task. When a new task is started, the old one is ended automatically.
	\item [Pipeline] marks the boundaries of a section of code that can be executed in parallel using a pipeline architecture.
	\item [PipelineIteration] represents one iteration of the pipeline.
	\item [PipelineSection] is a section of a iteration. These sections can be run in parallel to different sections from different iterations.
\end{itemize}

\subsubsection{Controlling tracing}
Filtering can reduce the amount of data gathered during a programs execution, but tags can provide a even more flexible approach. By default calls are recorded and memory accesses are ignored, but this can be changed with the appropriate tags.

\begin{itemize}
	\item [IgnoreAll] stops all tracing.
	\item [IgnoreCalls] stops the tracing of calls.
	\item [IgnoreAccesses] stops the tracing of accesses.
	\item [ProcessAll] forces the tracing of everything.
	\item [ProcessCalls] forces the tracing of calls.
	\item [ProcessAccesses] starts the tracing of accesses.
\end{itemize}

\begin{figure}
	\begin{center}
		\begin{minted}{c}
			if (ignoreCalls)
				processCallsComputed = false;
			else if (processCalls)
				processCallsComputed = true;
			else if (interestingProgramPart)
				processCallsComputed = true;
			else
				processCallsComputed = true;
			
			if (!processCallsComputed)
				processAccessesComputed = false;
			else if (ignoreAccesses)
				processAccessesComputed = false;
			else if (processAccesses)
				processAccessesComputed = true;
			else if (interestingProgramPart)
				processAccessesComputed = true;
			else
				processAccessesComputed = false;
		\end{minted}
	\end{center}
	\caption{Algorithm to determine if tracing is performed}
	\label{cap2:tracealg}
\end{figure}

\section{Tools}

\subsection{Database layout}

\subsection{pintool\_static.so}
\subsubsection{Source Locations}
\subsubsection{File View}
\subsubsection{Tagging}

\subsection{pintool\_dynamic.so}
\subsubsection{Architecture}
\subsubsection{Fast buffer API}
\subsubsection{Allocation interception}
\subsubsection{Shadow Stack}
\subsubsection{Tagging implementation}
\subsubsection{Memory accesses}
\subsubsection{Reference resolution}
\subsubsection{Thread handling}
\subsubsection{Tag handling}

\section{Analysis}

\subsection{Section}
\subsubsection{Introduction}
\subsubsection{Parallelization}
\subsubsection{Dependency detection}
\subsubsection{Visualization}

\subsection{Pipeline}
\subsubsection{Introduction}
\subsubsection{Parallelization}
\subsubsection{Dependency detection}
\subsubsection{Visualization}

\subsection{Calling Context Tree}
\subsubsection{Parceive UI}
\subsubsection{Modifications}





